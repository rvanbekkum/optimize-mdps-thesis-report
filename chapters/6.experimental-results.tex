\chapter{Experimental Setup and Results}
\label{ch:experimental-results}

\begin{itemize}
	\item Introduction about application and high-level explanation of what we'll be looking into
\end{itemize}

\section{Setup}
\label{sec:setup}

\begin{itemize}
	\item Software used for the implementation:
	\begin{itemize}
		\item Simulator: \textit{Morse}
		\item Programming Language: \textit{Python (2.7)} together with \texttt{scikit-learn}, \texttt{pymdptoolbox}, \texttt{bayesian-optimization} packages + explain what each of them are used for
		\item Control movements of robot in simulator through \textit{ROS}
	\end{itemize}
	\item Scitos-A5 robot, mobile service robot (plus \textit{short} discussion of what this robot has been used for in the real world)
	\item Details on the implementation; how the framework/routine is implemented for this application
	\begin{itemize}
		\item 
	\end{itemize}
	\item Discuss the different configurations that are to be compared
	\begin{itemize}
		\item Fixed values for $\beta$-parameter vs. gradually decreasing from $1$ to $0$
		\item Multiple environments: small (\texttt{tum\_kitchen}); large (\texttt{uol\_bl})
		\item Model-learning algorithms: direct clustering vs. trajectory clustering; $k$-means vs. GMM
		\item Data-sets of different sizes
	\end{itemize}
\end{itemize}

\section{Results}
\label{sec:results}

\begin{itemize}
	\item First show results of optimization only based on simulation outcomes (that is: $\beta = 1.0$)
	\item Then show results for the cost-incremental extension of the framework
\end{itemize}

