\chapter{Experimental Setup and Results}
\label{ch:experimental-results}

%\begin{itemize}
%	\item Introduction about application and high-level explanation of what we'll be looking into
%\end{itemize}

In \autoref{ch:methodology} a framework was proposed for finding an optimal \acrshort{acr:mdp} for planning problems that involve uncertainty given a dataset describing the dynamics of the system under consideration.
The framework aims to achieve this by posing the adjustment of the parameters of model learning algorithms as an optimization task in which the yielded performance is to be maximized.
The domain of mobile robot navigation, where the problem statement is to navigate a robot between locations as fast as possible, was identified as a potentially suitable application.
This chapter discusses the experiments that were conducted to evaluate the framework for this application and the results that were obtained accordingly.
First of all, \autoref{sec:setup} elaborates upon the setup for the experiments, discussing the relevant details on the implementation of the framework for this application and the software and other resources used.
Then

\section{Setup}
\label{sec:setup}


%\begin{itemize}
%	\item Software used for the implementation:
%	\begin{itemize}
%		\item Simulator: \textit{Morse}
%		\item Programming Language: \textit{Python (2.7)} together with \texttt{scikit-learn}, \texttt{pymdptoolbox}, \texttt{bayesian-optimization} packages + explain what each of them are used for
%		\item Control movements of robot in simulator through \textit{ROS}
%	\end{itemize}
%	\item Scitos-A5 robot, mobile service robot (plus \textit{short} discussion of what this robot has been used for in the real world)
%	\item Details on the implementation; how the framework/routine is implemented for this application
%	\begin{itemize}
%		\item 
%	\end{itemize}
%	\item Discuss the different configurations that are to be compared
%	\begin{itemize}
%		\item Fixed values for $\beta$-parameter vs. gradually decreasing from $1$ to $0$
%		\item Multiple environments: small (\texttt{tum\_kitchen}); large (\texttt{uol\_bl})
%		\item Model-learning algorithms: direct clustering vs. trajectory clustering; $k$-means vs. GMM
%		\item Data-sets of different sizes
%	\end{itemize}
%\end{itemize}

\subsection{Software}
\label{sec:software}

Software used for the implementation:
\begin{itemize}
	\item Programming Language: \textit{Python (2.7)} together with \texttt{scikit-learn}, \texttt{pymdptoolbox}, \texttt{bayesian-optimization} packages + explain what each of them are used for
	\item Simulator: \textit{Morse}
	\item Scitos-A5 robot, mobile service robot (plus \textit{short} discussion of what this robot has been used for in the real world)
	\item Control movements of robot in simulator through \textit{ROS}
\end{itemize}

\subsection{Implementation}
\label{sec:implementation}

Details on the implementation; how the framework/routine is implemented for this application
\begin{itemize}
	\item Environments
	\item Exploration / Data Gathering
	\item Optimization
	\item Simulations of following policy
\end{itemize}

\section{Scenarios}
\label{sec:scenarios}

Discuss the different configurations that are to be compared
\begin{itemize}
	\item Fixed values for $\beta$-parameter vs. gradually decreasing from $1$ to $0$
	\item Multiple environments: small (\texttt{tum\_kitchen}); large (\texttt{uol\_bl})
	\item Model-learning algorithms: direct clustering vs. trajectory clustering; $k$-means vs. GMM
	\item Data-sets of different sizes
\end{itemize}

\vspace{12pt}
\noindent\fbox{\textbf{TODO:} Table to be added describing the configurations/scenarios for the experiments.}

\section{Results}
\label{sec:results}

\begin{itemize}
	\item First show results of optimization only based on simulation outcomes (that is: $\beta = 1.0$)
	\item Then show results for the cost-incremental extension of the framework
\end{itemize}

