\chapter{Introduction}
\label{ch:introduction}

% 

\section{Motivation}
\label{sec:motivation}

There are several practical applications in which the (sequential) actions of systems are coordinated by decision makers or \textit{agents} to achieve long-term goals.
In particular these agents need to take into account the uncertainty in these systems that may be present in the form of action failures (e.g., a robot slipping), exogenous events (e.g., moving obstacles) and noisy observations.
To devise optimal plans for those systems whose dynamics are stochastic, \acrlong{acr:dtp} aims to account for uncertainty by exploiting the considerable structure these systems pose through the development of probabilistic models which reflect this uncertainty.
These probabilistic models serve as a system representation which describe a system's state and its evolution over time after a sequence of actions has been executed.
The advantage of having such (accurate) probabilistic models at one's disposal is that agents can act according to different policies derived from one and the same model to perform multiple tasks.

In recent years particularly \acrfullpl{acr:mdp} have become a significant popular formalism for modeling \acrshort{acr:dtp} problems. 
That is, first of all, due to their firm foundation in decision theory and successes of Markovian approaches in speech recognition and the closely-related field of \acrfull{acr:rl}.
Furthermore, over the years various computationally efficient solution techniques have been devised for obtaining optimal plans for \acrshort{acr:mdp} models which maximize expected value.
Considering the expediency of \acrshortpl{acr:mdp} for automated control, it seems worthwhile to investigate methods of developing accurate probabilistic models for the purpose of planning under uncertainty.

% New paragraph here

% Mainly applied approach in automated control involves hand-crafting a mathematical model as a system representation that can be used to represent the system's state and its evolution over time after a sequence of actions has been executed.
% Setting up such models is an important, difficult and time-costly process for human designer that might not always have (enough) expertise to craft a suitable model that can be used to plan for the execution of tasks.

\section{Problem Description}
\label{sec:problem-description}

The problem this thesis is concerned with is the development of probabilistic models, which are used for obtaining plans for automated control of systems whose dynamics are stochastic.
Setting up such models is an important, though difficult and time-costly process for a human designer that might not always have (enough) expertise to craft a suitable, well-generalizing model that can be used to plan for the execution of different tasks.
In practice, this problem is tackled by domain experts by iteratively tweaking the model parameters until the desired performance is achieved.

An alternative that one ought to consider, is that of applying \acrshort{acr:rl} techniques rather than planning algorithms.
However, although ideas from planning and \acrshort{acr:rl} are interchangeable, these techniques typically demand direct interaction with the environment which is something that cannot always be readily offered.
That is, \acrshort{acr:rl} techniques turn out as time-consuming and sometimes even dangerous when applied in real-world environments.
Those domains for which plans need to be formulated offline, demand the acquisition of probabilistic models which accurately defines what constitutes the state of the system, reflect uncertainty through transition probabilities and some manner of specifying goals.

An attractive approach of bypassing the daunting and error-prone task of handcrafting probabilistic models, is that of automatizing the model development process by applying learning algorithms on data that describes the dynamics of the system.
For this approach the data is typically presented in the form of execution traces of the system operating in a real-world environment, which is collected in an exploration phase prior to the actual planning.
However, although one could obtain a model of the system by applying learning algorithms, the corresponding plans that are inferred from this model might not be effective when applied in a real-world environment.
First of all, this might be due to the parameters of the learning algorithm not being set properly, signifying the need for proper adjustment of these parameters.
Another possibility is that the gathered data is incomplete and so does not accurately describe the dynamics of the system.
In this case, one could choose to augment the data for those areas for which the training data is inadequate, although one should be aware that this is accompanied by a more cost-expensive model-learning process.
Therefore, in order to learn accurate probabilistic models, we are in need of a way of assessing the performance of a model accompanied by a method for identifying inadequacies due to incomplete data, while taking into account the corresponding cost of learning and evaluating these system models.

\section{Contribution}
\label{sec:contribution}

We propose a framework for automating the development of probabilistic models in the form of \acrshortpl{acr:mdp} by applying learning algorithms on data describing the stochastics of the system under consideration.
The data about the environment that is used by these learning algorithms is gathered prior to the model learning process in an exploration phase.
In order to obtain accurate probabilistic models for the system under consideration, the parameters of the learning algorithm should be tweaked in such way to optimize the performance in the execution of its tasks.
To achieve this we pose the adjustment of the learning algorithm parameters as an optimization task to maximize the performance that follows from executing plans derived from learned \acrshort{acr:mdp} models.
The optimization is then done by applying a technique known as Bayesian Optimization, so that we define a probability distribution over functions to model the performance measure and iteratively sample parameter-settings towards a global maximizer of the performance.

To address model inadequacies that are due to incomplete data, we propose to incorporate the identification of insufficient exploration for the transitions from those states that are visited in the execution of obtained plans.
That is, for those learned models for which the data was observed to be incomplete, the observations of the performance for the corresponding parameter-settings are modeled as noisy or uncertain.
This serves the purpose of more cautious exploitation of those areas of the parameter space which mostly produce noisy observations of the performance measure.

The framework is applied, tested and evaluated for the domain of mobile robot navigation, as the robots in this domain oft to operate under significant uncertainty in their actions.
A solution in this context is a policy which maps discretized robot poses into fine-grained navigation actions so to move a mobile robot to a certain goal location.
Probabilistic models are acquired by applying unsupervised machine learning algorithms on execution traces (consisting of odometry data describing robot poses) obtained in an exploration phase.
The optimization of the model is based on the performance of acquired models in simulations, which is expressed in terms of the time that has passed to reach multiple goal locations.
Incomplete exploration data is here identified by inspecting the trajectories that are obtained from the execution of an optimal plan for an \acrshort{acr:mdp}.

% labeled as noisy uncertain observation of the performance of the model

% Discuss the learning and optimization routine: Automized method for learning probabilistic models in the form of MDPs by applying (machine) learning algorithms on exploration data. Optimization of the model by Bayesian Optimization based on performance in simulations.
% Identify incomplete exploration data about the environment by inspecting the observed trajectories and provide feedback on this from the optimization loop.
% Application to mobile robot navigation; A solution in this context is a policy that maps discretized robot poses into fine-grained navigation actions.

\section{Scope and Limitations}
\label{sec:scope-limitations}

First of all, we restrict ourselves to learning fully observable \acrshortpl{acr:mdp} from data.
However, this framework could possibly be extended to deal with applications where the states cannot be directly observed and should be inferred from observations.

Secondly, we note that all experiments and corresponding results are based on data that is solely obtained from simulations of a mobile robot.
Reasons for this are that we make the assumption that the environment is fully observable, which is typically not the case for real-world applications, and not having a real robot at hand, which would be too expensive and comes with its own technical challenges that are outside of our scope.

Finally, we will discuss how the proposed framework could be applied to other domains, although we will focus on the domain of mobile robot navigation to evaluate and test our approach.
One of the important aspects to consider, is which data should be collected to describe the stochastics of the system to learn models from this data.
For our implementation we choose to collect data about the robot's poses from internal odometry, but for other applications we might not be able to describe the states and transitions of the system through a geometric model and may need other learning algorithms than the clustering algorithms we use for our application.
% TODO

% POMDPs: Restricted to MDPs, but could possible be extended to deal where the states are cannot be directly observed and should be inferred from observations.
% Field of application: We will discuss how the proposed method could be applied to other domains, although we will focus on the domain of mobile robot navigation to evaluate and test our method.

\section{Overview}
\label{sec:introduction-overview}

The framework that we propose is aimed at automating the development of probabilistic models for the purpose of automated control by agents through planning under uncertainty.
In the learning routine the plans are obtained through techniques founded in \acrlong{acr:dtp} in which probabilistic models are employed to achieve this.
In \autoref{ch:background} we make the type of problems that are faced in \acrshort{acr:dtp} concrete and review how specialized probabilistic models can be used by existing algorithms to acquire plans.
For the optimization of the parameters of learning algorithms we use a method known as Bayesian Optimization. In \autoref{ch:bayesian-optimization} we formally describe how this method works and how it is relevant to sequential decision making, supported by a number of successful applications for learning and planning.
In \autoref{ch:problem-related-work} the problem we aim to solve is defined in more detail and explore the existing model learning algorithms as described in related work.
The proposed solution to this problem is explained in \autoref{ch:setup}, in which we put the theory and algorithms discussed in the earlier chapters together into an optimization routine.
To test and evaluate our solution we choose to make an implementation for the optimization of probabilistic models for the path planning of a mobile robot.
In \autoref{ch:experimental-results} we discuss the results obtained from testing the proposed solution with our implementation for mobile robot navigation.
Finally, in \autoref{ch:conclusions} we summarize and evaluate the proposed solution.

%TODO Update this overview (e.g., Chapter 4 only presents related work; other chapters may change and we should describe (really) short the contents)