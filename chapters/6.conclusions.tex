\chapter{Conclusions}
\label{ch:conclusions}

The goal of this thesis is to provide a foundation for an algorithmic technique for learning \acrfullpl{acr:mdp} for planning under uncertainty which maximize yielded performance given a dataset describing the dynamics of a system that involves uncertainty.
This chapter presents a summary of the presented work and the contributions made in \autoref{sec:summary} and revisits the identified research questions in \autoref{sec:revisiting-research-questions}.
Finally, this chapter concludes this thesis with recommendations and suggestions for future work in \autoref{sec:recommendations-future-work}.

\section{Summary of Contributions}
\label{sec:summary}

Previous work in the field of probabilistic model learning for planning under uncertainty has already presented us with various algorithms for learning \acrfullpl{acr:mdp} offline from a dataset describing the dynamics of the system under consideration.
The state of the art however lacks an automated method of setting the hyperparameters of these learning algorithms so to best reflect the underlying system and maximize its performance in the execution of the tasks it is expected to perform.
To address this issue, we pose the adjustment of the hyperparameters of such learning algorithms as an optimization task.
In this optimization task, the goal is to find the setting of the hyperparameters $\theta$ which maximizes the performance yielded by executing the plans or \textit{policies} derived from the learned~\acrshort{acr:mdp}.

In this thesis, we present a solution (which we refer to as the \textit{\acrshort{acr:mdp} optimization framework}) that employs the \acrfull{acr:bo} framework for this optimization task, defining a probability distribution over functions which maps parameter settings $\theta$ to an assessment of the value $V_\mathcal{M}$ of a learned \acrshort{acr:mdp}.
Although algorithms that employ the \acrshort{acr:bo} framework for \acrshort{acr:sdm} problems do already exist, all of them are online \acrshort{acr:rl} approaches that do not utilize an available dataset prior to interacting with the real-world environment.
The model value $V_\mathcal{M}$ of an \acrshort{acr:mdp}, used as a relative performance measure in the optimization, is assessed based on computed value functions and simulations of the tasks the system is expected to perform.

Additionally, we extended our proposed solution by exploiting the lower cost of computing a value function in comparison to performing time-expensive simulations, to define our \textit{multi-phase optimization framework}.
That is, we define a first phase in which \acrshort{acr:bo} is performed to maximize the average expected value for the \acrshort{acr:mdp} for a set of tasks to perform.
The posterior resulting from this phase is then used to steer the acquisition in a second optimization phase, in which the performance in simulations over a set of tasks is being optimized.

Then, finally, we present a solution to further fine-tune the parameters of the \acrshort{acr:mdp} resulting from this optimization.
This is done by increasing the resolution of the state space and updating the transition probabilities where necessary, e.g., when the agent gets stuck in a certain state.

An implementation of the aforementioned framework has been made for path planning in a mobile robot navigation domain, in which a robot needs to move from one location to another in an office environment.
A dataset of execution traces with odometric readings has been gathered based on which our implementation learns \acrshortpl{acr:mdp} by clustering the data into a state~space, with the number of states defined by the parameter setting $\theta$.
The implementation thus optimizes for an \acrshort{acr:mdp} that can produce policies for a mobile robot offline, which can be followed to move from one location to another as fast as possible.

The results of the experiments show that our framework can effectively be used to obtain an \acrshort{acr:mdp} for the path planning of a mobile robot.
The first phase in the multi-phase framework shows out to be able to steer the acquisition in the second phase towards a global maximizer in some scenarios.
And finally, the proof-of-concept implementation of the last phase, can successfully be used to further fine-tune the parameters of a learned \acrshort{acr:mdp} when the mobile robot gets stuck in some state when presented with a new task.
However, it should be noted that the last phase only has been tested on a single task (considering it is just an extra step on top to potentially further improve \acrshortpl{acr:mdp}), which means we cannot simply guarantee it can be used to recover from all discrepancies that may be present.

% Discussion of the results
% Contributions made

\section{Revisiting the Research Questions}
\label{sec:revisiting-research-questions}

To answer the main research question of this thesis, the four research questions presented in \autoref{sec:research-questions} are revisited in this section.
First off, the following research question is mostly concerned with getting a good overview of the state of the art for learning \acrshortpl{acr:mdp} from a dataset:

\vspace{16pt}
\noindent%\fbox{\parbox{\textwidth}{
\textbf{Research Question 1.} Which learning algorithms exist that can be employed for learning \acrshortpl{acr:mdp} from data for systems involving uncertainty that require plans for automated control?
%}}
\vspace{0pt}

As seen in \autoref{sec:learning-markov-models}, one of the most straightforward methods can be deducted from methods for fitting Markov Chains, i.e. by maximum likelihood or Bayesian inference.
The difference is in the addition of actions, so that a transition probability matrix needs to be learned for each possible action, given some user-defined state space.
For various domains, such as that of mobile robot navigation, defining the state space may not be a trivial task, although there are approaches for this, such as $k$-Means clustering or (time-)state merging approaches, which are parameterized by the number of states.
Then, when one needs to learn partially observable models one needs to consider other approaches, as seen in \autoref{sec:learning-probabilistic-models}, to account for emission probabilities as well.

\vspace{16pt}
\noindent%\fbox{\parbox{\textwidth}{
\textbf{Research Question 2.} How should a performance measure be defined which can be used to fairly compare the value of different \acrshortpl{acr:mdp}?
%}}
\vspace{12pt}

The value $V_\mathcal{M}$ of an \acrshort{acr:mdp} $\mathcal{M}$ should be defined in terms of how well the agent performs tasks based on the policies computed from the model.
Therefore, first of all, the performance can be estimated using the expected value in the initial state from the value function for multiple tasks.
However, as a model abstracts from the real world, using the value function to express model value is not always sufficient.
Therefore, a more accurate estimation can be made through simulations of the tasks the system is expected to perform, discounting the obtained reward based on the number of steps made.
Combining these estimations results in the expression shown in \autoref{eq:vm} in \autoref{sec:learning-step}, which can be used as a relative performance measure of different \acrshortpl{acr:mdp} for an environment.

\vspace{16pt}
\noindent%\fbox{\parbox{\textwidth}{
\textbf{Research Question 3.} How can the parameter space of model learning algorithms cost-effectively be explored towards a global maximizer with only limited knowledge about the system under consideration?
%}}
\vspace{12pt}

We have seen that making an accurate assessment of the value of an \acrshort{acr:mdp} requires time-expensive simulations to be performed.
Therefore, it is important to limit the number of evaluations of a parameter setting for the used model learning algorithm, with \acrshort{acr:bo} emerging as an attractive framework for optimization.
As we have seen in our experiments, \acrshort{acr:bo} could effectively be employed to explore the parameter space with a limited number of evaluations.

\newpage

\vspace{16pt}
\noindent%\fbox{\parbox{\textwidth}{
\textbf{Research Question 4.} How can the hierarchy of different abstraction levels be exploited to find a performance-maximizing \acrshort{acr:mdp} in a more cost-effective way?
%}}
\vspace{12pt}

This research question has been addressed by the multi-phase framework proposed in \autoref{sec:multi-phase-framework}.
In the first phase \acrshort{acr:bo} for an \acrshort{acr:mdp} with the model value based solely on the value functions for the set of tasks over which to evaluate the performance.
The resulting posterior is then used in the second phase to steer the acquisition of the first few samples.
In our experiments we have seen that it is possible to use the first phase to successfully steer the optimization in the second phase towards a global maximizer.
\vspace{12pt}

Putting our findings for our sub-questions together, we provide an answer to our main research question reiterated below.

\vspace{12pt}
\noindent%\fbox{\parbox{\textwidth}{ %Uncomment this for boxes
\textbf{Main Research Question.} How can the task of obtaining a (discrete) \acrshort{acr:mdp} that maximizes the yielded performance of executing plans that are derived from it, given a dataset about the system under consideration, be automated?
%}}
\vspace{12pt}

As handcrafting \acrshortpl{acr:mdp} for systems with uncertain dynamics is a difficult task, an appealing approach is to employ learning algorithms to automate this task.
To learn an \acrshort{acr:mdp} which maximizes performance yielded from following the policies computed from it, one needs to properly adjust the hyperparameters of these algorithms.
As evaluating all possible hyperparameter settings may be a cost-expensive endeavor, the parameter space of model learning algorithms may be explored more effectively using the sequential model-based optimization framework known as \acrfull{acr:bo}.
To compare \acrshortpl{acr:mdp}, assessments of their value can be made at different levels of abstraction, i.e., based solely on the \acrshort{acr:mdp} and its value function, based on performance in simulations, or based on performance in a real-world environment.
Additionally, it was found that assessments on a higher level of abstraction (with lower costs) can be employed to steer the optimization on a lower level of abstraction towards the area of the parameter space that maximizes the performance.
However, we should note that to give a conclusive answer to our main research question, we need to investigate how well our methodology works for different application domains.

\section{Recommendations and Future Work}
\label{sec:recommendations-future-work}

As the framework proposed in this thesis is targeted at offline planning, the policies computed from resulting \acrshortpl{acr:mdp} can only to a certain extent account for the exogenous events in the environment.
Suppose an agent for a mobile robot, like in our example application, is employed in the real world and at some point the planned path is obstructed.
In this scenario the agent will be unable to get to its target location based \textit{solely} on the policy that was computed offline.
To account for these scenarios, first off, one could choose to periodically explore the environment (using the current model as a blueprint, as in \cite{epshteyn2008active}) and update the model parameters accordingly instead of having just a single exploration phase.
An alternative is to employ the learned \acrshort{acr:mdp} in one of the model-based \acrshort{acr:rl} methods seen in \autoref{ch:problem-related-work}, which automatically balance exploration and exploitation.

Another issue that might be interesting to address in future work, is to identify whether or not sufficient data is available for learning an appropriate \acrshort{acr:mdp} model.
Although in our experiments we were able to observe a high level of noise and overfitting on smaller datasets, still the problem exists of how to assess what dataset size is appropriate and whether or not the dataset covers the complete environment.

Considering the optimization aspect of the presented framework, one of the issues that were identified, is that selecting the acquisition function and the hyperparameters of the \acrshort{acr:gp} prior for \acrshort{acr:bo} is not a trivial task.
For future work, we recommend the use of the portfolio allocation \cite{Hoffman2011, shahriari2016taking} and integrated acquisition \cite{snoek2012practical} functions, discussed in \autoref{ch:background}, which ease the above task as they reduce the number of choices that need to be made.

\newpage

Probably one of the most interesting directions for future work is to examine and experiment with model learning through our framework for different domains.
The main question then is whether or not the application domain under consideration can provide a dataset that describes the dynamics of the system and allows for learning probabilistic models.
Considering that \acrshortpl{acr:mdp} and especially \acrshortpl{acr:pomdp} are data intensive, one should carefully contemplate the contents of the dataset and whether they allow for learning transition probabilities, emission probabilities, state spaces and/or observation spaces which accurately reflect the dynamics of the underlying system.
One recommendation is to look into learning \acrshortpl{acr:pomdp} for spoken dialogue systems \cite{chinaei2011, png2011bayesian, young2016}.
Recently, there has been much interest in modeling the dialogue manager of these systems this way, although estimating their dynamics is quite a difficult task, and so it is desirable to automate this process.
As an example, in \cite{chinaei2011}, the \acrshort{acr:pomdp}'s components (i.e., state space, observation space, transition and emission probabilities) are learned based on a dataset of human-to-human dialogues.
For such domains, our framework may prove useful in learning a probabilistic model that maximizes the performance of the system.
For other potential application domains, one may want to consult \cite{cassandra1998spa} which presents a variety of applications for \acrshortpl{acr:pomdp} which may potentially be learned from data.

Another topic we briefly touched upon in \autoref{ch:background} is \textit{automated model checking}, which encompasses techniques for formal model verification through temporal logic formulas.
Examples can be found in \cite{bhatia2010sampling, lacerda2015optimal} where \acrshort{acr:ltl} formulas are used to formally express temporal goals and produce a `product-\acrshort{acr:mdp}' to derive plans which (partially) satisfy these formulas.
An interesting direction for future work would be to investigate the possibilities of combining such model checking techniques with our framework.
That is, combining the methods from automated model checking and model learning through our framework, one may be able to develop a framework that optimizes for an \acrshort{acr:mdp} which is best capable of satisfying a collection of \acrshort{acr:ltl} formulas.

%As this thesis has only limited itself to learning and optimizing fully observable \acrshortpl{acr:mdp}, one direction for proposed future work is to explore the possibilities of optimizing for \acrshortpl{acr:pomdp}.
%That is, we should acknowledge that our experiments make the simplifying assumption that the state is directly observable from odometric readings, while in practice this does not hold up and would require execution traces with other sensor readings.
%This would require one to make use of other model learning algorithms, such as the \textit{Baum-Welch} algorithm and state merging algorithms seen in \autoref{sec:learning-probabilistic-models}.
%Apart from this, another recommendation for future work would be to implement and test the solution on applications other than path planning for mobile robot navigation.
%Further, one could consider to explore and investigate alternative global optimization algorithms (e.g., see \cite{bergstra2011algorithms}) to address the problem statement of this thesis.

%\vspace{12pt}
%\noindent\minibox[frame]{\textbf{Note:} This section will be updated and extended for the final version of the report.}

% Recommendation: POMDPs for Dialogue Management as seen in http://mi.eng.cam.ac.uk/~sjy/papers/ygtw13.pdf
% http://citeseerx.ist.psu.edu/viewdoc/download?doi=10.1.1.129.2652&rep=rep1&type=pdf
% Model checking, for instance refer again to lacerda2015 and refer to the routine in figure shown in earlier chapter
