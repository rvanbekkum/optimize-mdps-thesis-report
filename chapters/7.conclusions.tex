\chapter{Conclusions}
\label{ch:conclusions}

The goal of this thesis is to provide a foundation for an algorithmic technique for learning \acrfullpl{acr:mdp} which maximize yielded performance given a dataset describing the dynamics of the system under consideration.
This chapter concludes this thesis with a summary of the presented work (in \autoref{sec:summary}) and revisits the identified research questions (in \autoref{sec:revisiting-research-questions}).
Finally, this chapter concludes this thesis with suggestions for future work (in \autoref{sec:recommendations-future-work}) and our concluding remarks (in \autoref{sec:concluding-remarks}).

% 'Don't ignore RL, could be used together'
% Continuous-state?
% Partially Observable?

\section{Summary}
\label{sec:summary}

% Discussion of the results
% Contributions made

\section{Revisitation of Research Questions}
\label{sec:revisiting-research-questions}

% Revisiting the research questions

\section{Recommendations and Future Work}
\label{sec:recommendations-future-work}

% 

\section{Concluding Remarks}
\label{sec:concluding-remarks}

%